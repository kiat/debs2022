%% The first command in your LaTeX source must be the \documentclass command.
\documentclass[sigconf]{acmart}

%%
%% \BibTeX command to typeset BibTeX logo in the docs
\AtBeginDocument{%
  \providecommand\BibTeX{{%
    \normalfont B\kern-0.5em{\scshape i\kern-0.25em b}\kern-0.8em\TeX}}}

%% Rights management information.  This information is sent to you
%% when you complete the rights form.  These commands have SAMPLE
%% values in them; it is your responsibility as an author to replace
%% the commands and values with those provided to you when you
%% complete the rights form.
\setcopyright{acmcopyright}
\copyrightyear{2022}
\acmYear{2022}
\acmDOI{XXXXXXX.XXXXXXX}

%% These commands are for a PROCEEDINGS abstract or paper.
% \acmConference[Conference acronym 'XX]{Make sure to enter the correct
%   conference title from your rights confirmation emai}{June 03--05,
%   2018}{Woodstock, NY}
% \acmPrice{15.00}
% \acmISBN{978-1-4503-XXXX-X/18/06}


%%
%% Submission ID.
%% Use this when submitting an article to a sponsored event. You'll
%% receive a unique submission ID from the organizers
%% of the event, and this ID should be used as the parameter to this command.
%%\acmSubmissionID{123-A56-BU3}

%%
%% The majority of ACM publications use numbered citations and
%% references.  The command \citestyle{authoryear} switches to the
%% "author year" style.
%%
%% If you are preparing content for an event
%% sponsored by ACM SIGGRAPH, you must use the "author year" style of
%% citations and references.
%% Uncommenting
%% the next command will enable that style.
%%\citestyle{acmauthoryear}

%%
%% end of the preamble, start of the body of the document source.
\begin{document}

%%
%% The "title" command has an optional parameter,
%% allowing the author to define a "short title" to be used in page headers.
\title{A High Performance Stream Processing System Implementation for Monitoring Stock Market Data Stream}

%%
%% The "author" command and its associated commands are used to define
%% the authors and their affiliations.
%% Of note is the shared affiliation of the first two authors, and the
%% "authornote" and "authornotemark" commands
%% used to denote shared contribution to the research.
% \author{Ben Trovato}
% \authornote{Both authors contributed equally to this research.}
% \email{trovato@corporation.com}
% \orcid{1234-5678-9012}
% \author{G.K.M. Tobin}
% \authornotemark[1]
% \email{webmaster@marysville-ohio.com}
% \affiliation{%
%   \institution{Institute for Clarity in Documentation}
%   \streetaddress{P.O. Box 1212}
%   \city{Dublin}
%   \state{Ohio}
%   \country{USA}
%   \postcode{43017-6221}
% }

% \author{Lars Th{\o}rv{\"a}ld}
% \affiliation{%
%   \institution{The Th{\o}rv{\"a}ld Group}
%   \streetaddress{1 Th{\o}rv{\"a}ld Circle}
%   \city{Hekla}
%   \country{Iceland}}
% \email{larst@affiliation.org}

% \author{Valerie B\'eranger}
% \affiliation{%
%   \institution{Inria Paris-Rocquencourt}
%   \city{Rocquencourt}
%   \country{France}
% }

% \author{Aparna Patel}
% \affiliation{%
%  \institution{Rajiv Gandhi University}
%  \streetaddress{Rono-Hills}
%  \city{Doimukh}
%  \state{Arunachal Pradesh}
%  \country{India}}

% \author{Huifen Chan}
% \affiliation{%
%   \institution{Tsinghua University}
%   \streetaddress{30 Shuangqing Rd}
%   \city{Haidian Qu}
%   \state{Beijing Shi}
%   \country{China}}

% \author{Charles Palmer}
% \affiliation{%
%   \institution{Palmer Research Laboratories}
%   \streetaddress{8600 Datapoint Drive}
%   \city{San Antonio}
%   \state{Texas}
%   \country{USA}
%   \postcode{78229}}
% \email{cpalmer@prl.com}

% \author{John Smith}
% \affiliation{%
%   \institution{The Th{\o}rv{\"a}ld Group}
%   \streetaddress{1 Th{\o}rv{\"a}ld Circle}
%   \city{Hekla}
%   \country{Iceland}}
% \email{jsmith@affiliation.org}

\author{Kevin Li, David Klingler, Yuhan Gao, Jacob Rivera, Kia Teymourian}
\affiliation{
  \institution{The University of Texas at Austin}
  \city{Austin, TX}
  \country{USA}}
\email{{kevinali, davidklingler, yg6952, jacobrivera}@utexas.edu, kiat@cs.utexas.edu}

%%
%% By default, the full list of authors will be used in the page
%% headers. Often, this list is too long, and will overlap
%% other information printed in the page headers. This command allows
%% the author to define a more concise list
%% of authors' names for this purpose.
\renewcommand{\shortauthors}{Trovato and Tobin, et al.}

%%
%% The abstract is a short summary of the work to be presented in the
%% article.
\begin{abstract}
  High performance monitoring of stock market data stream is one of the challenging use cases of data stream processing system. 
  Different continuous queries can be defined by market brokers to provide buy and sell advices. 
  In this paper, we describe our implementation of a system for  DEBS2022 Grand Challenge to extract breakout patterns from real-time stock market data to provide buy and sell notifications. 
  Breakout patterns are specified based on long intervals of 50 days and 100 days known as the golden cross to indicate a golden opportunity for long-term investments.
  We report details of our high-performance implementation to extract such patterns in real-time be able to generate real-time buy and sell notifications.
\end{abstract}

%%
%% The code below is generated by the tool at http://dl.acm.org/ccs.cfm.
%% Please copy and paste the code instead of the example below.
% %%
% \begin{CCSXML}
% <ccs2012>
%  <concept>
%   <concept_id>10010520.10010553.10010562</concept_id>
%   <concept_desc>Computer systems organization~Embedded systems</concept_desc>
%   <concept_significance>500</concept_significance>
%  </concept>
%  <concept>
%   <concept_id>10010520.10010575.10010755</concept_id>
%   <concept_desc>Computer systems organization~Redundancy</concept_desc>
%   <concept_significance>300</concept_significance>
%  </concept>
%  <concept>
%   <concept_id>10010520.10010553.10010554</concept_id>
%   <concept_desc>Computer systems organization~Robotics</concept_desc>
%   <concept_significance>100</concept_significance>
%  </concept>
%  <concept>
%   <concept_id>10003033.10003083.10003095</concept_id>
%   <concept_desc>Networks~Network reliability</concept_desc>
%   <concept_significance>100</concept_significance>
%  </concept>
% </ccs2012>
% \end{CCSXML}

% \ccsdesc[500]{Computer systems organization~Embedded systems}
% \ccsdesc[300]{Computer systems organization~Redundancy}
% \ccsdesc{Computer systems organization~Robotics}
% \ccsdesc[100]{Networks~Network reliability}

% %%
%% Keywords. The author(s) should pick words that accurately describe
%% the work being presented. Separate the keywords with commas.
% \keywords{datasets, neural networks, gaze detection, text tagging}

%% A "teaser" image appears between the author and affiliation
%% information and the body of the document, and typically spans the
%% page.
% \begin{teaserfigure}
%   \includegraphics[width=\textwidth]{sampleteaser}
%   \caption{Seattle Mariners at Spring Training, 2010.}
%   \Description{Enjoying the baseball game from the third-base
%   seats. Ichiro Suzuki preparing to bat.}
%   \label{fig:teaser}
% \end{teaserfigure}

\maketitle



\section{Introduction}
The trading of stock shares can be triggered by high complex event patterns that are specified
based on patterns in the event stream of the stock market. Complex patterns are are specified based on the history of 
the event stream, for example, by utilizing moving averages of stock prices and the trading volumes.
The real-time extraction of such complex patterns to trigger buy or sell trading actions is the task
of high-performance event stream processing systems.

% Describe briefly what the challenge is
This year's 2022 DEBS Grand Challenge \cite{debs2022challenge} describes a system implementation based on two specific queries on the
stock market event streams. The first query is defined to compute the Exponential Moving Average (EMA) with two different smoothing factors of 38 and 100.
An exponential Moving Average is one of the moving averages and is defined as follows:

\begin{align*}
    EMA_t = \begin{cases}
        Y_0 &  t = 0 \\
        \alpha Y_t + (1-\alpha) EMA_{t-1}& t>0 \\
        \end{cases}
\end{align*}

The coefficient $\alpha$ represents the degree of weighting decrease, a constant smoothing factor between 0 and 1.
For this challenge $\alpha = \frac{2}{1+j}$ where $j$ is a smoothing factor with $j \in \{38, 100 \}$.
We use $EMA_{38}$ to refer to the exponential moving average with a smoothing factor of 38 and $EMA_{100}$ for a smoothing factor of 100.


\begin{figure}[!ht]
    \begin{center}
        \includegraphics[width=0.42\textwidth]{./images/stock_example.png}
        \caption{An Example of Stock Price Fluctuations Over Time}
        \label{fig:stock}
    \end{center}
\end{figure}



\begin{figure*}[!ht]
    \begin{center}
        \includegraphics[width=0.70\textwidth]{./images/query2_example.png}
        \caption{Example of Query 2 - Buy and Sell advice based on Breakout Patterns of EMA 38 and 100 }
        \label{fig:EMAs}
    \end{center}
\end{figure*}


% Describe the query 1 and 2
Figure \ref{fig:stock} illustrates an example of stock market price changes over time. The graph shows 1000 events with
different prices. Figure \ref{fig:EMAs} depicts the values of the exponential moving average with
smoothing factors of 38 and 100. We can observe in Figure \ref{fig:EMAs} how the value grows exponentially
and how the two $EMA_{38}$ and $EMA_{100}$ are different from each other.

Query 2 of the DEBS 2022 challenge \cite{debs2022challenge} is specified based on the results of Query 1. Each time
that $EMA_{38}$ breaks out of the value of $EMA_{100}$ a stock buy advice notification should be generated and if
$EMA_{100}$ goes under the $EMA_{38}$ a stock sell advice. For a further detailed description of the DEBS 2022
Grand Challenge, we would refer the readers to \cite{debs2022challenge}.

Figure \ref{fig:EMA200} visualizes the same graph as we have in Figure \ref{fig:EMAs} by
zooming into the graph (range 0 to 400 events on x-axis) to see the values and their differentiations. One important observation in
Figures \ref{fig:EMA200} and \ref{fig:EMAs} is that $EMA_{38}$ and $EMA_{100}$ have a large difference in the first
200 events.  The reason for this difference is that the exponential moving average is specified based
on the history of events to increase the value exponentially and requires a warm-up phase. Based on this observation,
one can improve the performance of the Query 2 by skipping the first 200 events because
$EMA_{38}$ and $EMA_{100}$ still have a large difference.

The main system development challenge task is to design a system that can process the stream of events with high throughput and low latency.
Many open source and commercial stream processing systems are developed that one can use to develop this challenge.
Esper event stream processing system \cite{Bernhardt2007} is a system that can detect complex events based on pre-defined temporal logic patterns.
In this task we do not have a complex pattern to extract and computations are basic computations of EMA 38 and 100, following with a check of the values for
Query 2 to submit a sell or buy advice. Other systems like Apache Storm \cite{8288619}, Apache Spark Streaming \cite{zaharia2010spark} or
Apache Flink streaming \cite{alexandrov2014stratosphere} have been developed to achieve high-scalability in data stream processing.
Also, different benchmarks are developed \cite{8701904} which compare these systems with each other regarding specific data
stream processing tasks.

After considering all of the existing systems and their overhead trade-offs, we decided to implement the DEBS Grand Challenge from scratch,
without using any of the existing systems like Apache Spark or Flink. Most of these systems have a different target like higher scalability than higher performance, so that 
they have a large start-up delay and additional cluster management costs. 

% Our initial implementation is a prototype of the system in python to make sure that we understood the logic of the implementation challenge correctly. 
First, we implemented the two data stream queries as a prototype system in python and run tests to check if the data streams are processed correctly, and the correct
results are submitted to the DEBS 2022 evaluation system. In the second project phase we implemented the same system in Java to achieve better processing 
performance (a better performance can be achieved by using a system language like C++ or Rust).  
Our implementation in Python and Java are available on Github\footnote{\url{https://github.com/kiat/debs2022} last update June, 2022}


% High-Performance computing problems
% Scalability issues

\begin{figure}[!ht]
    \begin{center}
        \includegraphics[width=0.4\textwidth]{./images/query2_example_200.png}
        \caption{Exponential Moving Average of 38 and 100 for the first 400 events.}
        \label{fig:EMA200}
    \end{center}
\end{figure}


The next subsequent sections describe details of our implementation. Section \ref{sec:concepts} describes our conceptual design for
a stream processing system using multiple processing threads on a single machine with multiple CPU cores. Further we describe how the same
architecture can be extended to process the data stream on a cluster of machines (multi-processing).  Section \ref{sec:implementation} provides
a brief description of important implementation details and Section \ref{sec:experiments} provides a brief overview of example
experiment results.


\section{Implementation Concepts and Architecture}
% Describe different ideas for distribution of events on multiple threads/processing and machines.  


\begin{figure*}[!ht]
    \begin{center}
        \includegraphics[width=0.9\textwidth]{./images/Parallel-Stream-Processing-System}
        \label{fig:parallel-srream-processing}
        \caption{Distribution of Stream Processing Task using multiple Event Producer and Consumers to process Query 1 and Query 2 }
    \end{center}
\end{figure*}



\section{Implementation Details}






\section{Experiments and Evaluation}\label{sec:experiments}
% Describe our results based on different number of producer and consumer numbers. 


Experiments with different number of event producer threads and consumer threads. 

Experiment with different architectures. 




\section{Conclusion}




\bibliographystyle{ACM-Reference-Format}
\bibliography{references}


\end{document}
\endinput
